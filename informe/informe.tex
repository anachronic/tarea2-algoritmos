\documentclass[12pt,letterpaper]{report}

\usepackage[spanish]{babel}
\usepackage[utf8]{inputenc}
\usepackage[right=2cm,left=3cm,top=2cm,bottom=2cm,headsep=0cm,footskip=0.5cm]{geometry}
\usepackage{graphicx}
\usepackage{wrapfig}

\title{\Huge Diccionarios en Memoria Secundaria \\ Tarea 2 \\ CC4102 - Diseño y Análisis de Algoritmos}
\author{Nicolás Salas V.\\Daniel Rojas C.}
\def\thesection       {\arabic{section}}
\sloppy

\begin{document}

\pagestyle{empty}
\begin{figure}[t]
\includegraphics[scale=0.83]{logo.png}
%\hspace{3.5cm}
\begin{tabular}{l}
\small Universidad de Chile\\
\small Facultad de Ciencias Físicas y Matemáticas\\
\small Departamento de Ciencias de la Computación\\
\small CC4102 Diseño y Análisis de Algoritmos\\
\small Prof: Gonzalo Navarro B.
\vspace{2.3cm}
\end{tabular}
\end{figure}

\maketitle

\tableofcontents
\newpage

\section{Introducción}

Intro

\section{Hipótesis}
hipo

\section{Diseño Experimental}

Para entender mejor los resultados que se presentan en la siguiente sección, en esta sección se describen todos los detalles de implementación de las estructuras de datos que se usan en los experimentos.\\

Es natural que los resultados puedan variar mucho según la forma en la que se implementan las estructuras de datos en memoria secundaria, y es justamente por eso que leer esta sección es de especial importancia para entender los resultados de este experimento \emph{particular} con las implementaciones aquí descritas.

\subsection{Árbol B (B-Tree)}

\subsection{Hashing Extendible}
\subsection{Hashing Lineal}
\subsubsection{Política 1}
\subsubsection{Política 2}

\section{Resultados}

\section{Análisis e Interpretación de los Datos}

\section{Conclusiones}

\section{Anexos}

\end{document} 


































